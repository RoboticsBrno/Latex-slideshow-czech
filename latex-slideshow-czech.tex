\documentclass[14pt]{beamer} %beamer - typ/šablona prezentace

\usepackage[czech]{babel} %nastavuje české popisky např. u obsahu, referencí, tabulek, obázků 
\usepackage[utf8]{inputenc} %použito UTF8 kvůli češtině (zvládá prakticky všechny jazyky na světě)
\usepackage[T1]{fontenc}

\usepackage{lmodern}
%\usepackage{datetime}
\usepackage{amssymb} %podpůrná knihovna pro matematické symboly
\usepackage{enumerate} %umožňuje širší možnosti nastavení enumerate

\usepackage{graphicx} %vkládání obrázků
\usepackage{hologo} %logo BibTeX

% Themes: http://www.hartwork.org/beamer-theme-matrix/
\mode<presentation>{\usetheme{Madrid}}
\usecolortheme{beaver}
\beamertemplatenavigationsymbolsempty 
\setbeamertemplate{title page}[default][colsep=0bp,rounded=true]
\setbeamertemplate{itemize items}{-} %$\circ$
\setbeamercolor*{item}{fg=black}
\setbeamertemplate{enumerate item}[default]

\author{J. Páral, V. Boček}
\institute[paral@robotikabrno.cz]{Pobočka Robotárna - Dům dětí a mládeže Brno, Helceletova\\[0.5cm]}
\title{Jak a proč používat \LaTeX}


\begin{document}

\frame{\titlepage}

\begin{frame}
    \frametitle{Osnova prezentace}
    \begin{center}
		\begin{enumerate}[a)]
			\item Úvod
			\item Jak na to
			\item Výhody \LaTeX{\lower .5ex\hbox {U}}
			\item Texové slabiny
			\item Prezentace v \LaTeX{\lower .5ex\hbox {U}}
			\item A kam dál?
		\end{enumerate}
    \end{center}
\end{frame}


\begin{frame}
    \frametitle{Úvod}
    \begin{center}
		\begin{enumerate} %[<+->]
			\item Co je to \LaTeX %\LaTeX{\lower .5ex\hbox {U}\kern -.125emM}
				\begin{itemize}
					\item program pro počítačovou sazbu			
				\end{itemize}	
			\item Kdy a proč vznikl
				\begin{itemize}
					\item vznikl v 70. letech 20. století
					\item tvůrce Donald Ervin Knuth
					\item mnoho chyb ve skriptech :-)
				\end{itemize}			
			\item K čemu je primárně určen
				\begin{itemize}
					\item primárně vyvinut pro sázení vzorců
			\end{itemize}
			\item Kdy se vyplatí jej používat
				\begin{itemize}
					\item rozsáhlé a strukturované práce \\
					(ročníkovky, SOČky, bakálářky, diplomky,...)
				\end{itemize}
		\end{enumerate}
    \end{center}
\end{frame}

\begin{frame}
    \frametitle{Jak na to}
	Co potřebuji pro zprovoznění Texu:
    \begin{center}
		\begin{enumerate} %[<+->]
			\item Překládací software
				\begin{itemize}
					\item \href{https://miktex.org/}{TeX Live} [Windows, MacOsX, Linux]
					\item MiKTex [Windows]		
				\end{itemize}	
			\item Editor zdrojové kódu
				\begin{itemize}
					\item Texmaker [free - Windows, MacOsX, Linux]
					\item Atoms/Visual Studio Code [free - Windows, MacOsX, Linux]
					\item PSPad [free - Windows]
					\item Notepad++ [free - Windows]
				\end{itemize}			
		\end{enumerate}
    \end{center}
\end{frame}

\begin{frame}[fragile]
    \frametitle{Jak na to}
	Ukázka jednoduchého dokumentu v Texu:
	\begin{center}
	\begin{verbatim}
		\documentclass[12pt]{article}
		\begin{document}
		Hello world!
		$a^2+b^2=c^2$ %math mode 
		\end{document}
	\end{verbatim}
	\end{center}
	Pozor na češtinu!
	\verb|\usepackage[utf8]{inputenc}|
	
\end{frame}

\begin{frame}[fragile]
    \frametitle{Výhody Texu}
    \begin{center}
		\begin{enumerate}
			\item Jednotné formátování v dokumentu
			\item Sazba kapitol
			\item Generování obsahu
			\item Seznamy obrázků, vzorců, tabulek
			\item Citace
			\item Poznámky 
			\item Odkazy v textu
			\item Vkládání vzorců
			\item Číslování stran
			\item Přenositelnost
		\end{enumerate}
    \end{center}
\end{frame}

\begin{frame}[fragile]
    \frametitle{Výhody Texu}
    1. Jednotné formátování v dokumentu
    \begin{center}
		\begin{itemize}
			\item používá všude stejný font
			\item a již v základu jsou pěkné :-)
			\item nestaráte se o formátování nadpisů,\\
			odkazů, poznámek, citací
		\end{itemize}
    \end{center}
\end{frame}

\begin{frame}[fragile]
    \frametitle{Výhody Texu}
    2. Sazba kapitol
    	       
    \verb|\titulek[ text do obsahu ]{ nadpis }|
	\verb|\titulek*{ nadpis } nečíslovaný nadpis|
    
    \begin{verbatim}
		\section{Nadpis první úrovně}
		\subsection{Nadpis druhé úrovně}
		\subsubsection{Nadpis třetí úrovně}
    \end{verbatim}
\end{frame}

\begin{frame}[fragile]
    \frametitle{Výhody Texu}
    3. Generování obsahu
	
	Lze provést přidáním jednoho příkazu a následně se aktualizuje při každém přeložení. 

    \begin{verbatim}
		\tableofcontents
    \end{verbatim}	    

	\textit{Pozor, většinou je potřeba přeložit dokument dvakrát, pro zobrazení změn v obsahu. Nebo lze automatický volat překladač dvakrát.}
\end{frame}

\begin{frame}[fragile]
    \frametitle{Výhody Texu}
    4. Seznamy obrázků, vzorců, tabulek
	
	Platí to samé jako pro obsah. 

    \begin{verbatim}
		\listoffigures %seznam obrázků
		\listoftables %seznam tabulek
		
    \end{verbatim}	    

	\textit{Pozor, většinou je potřeba přeložit dokument dvakrát, pro zobrazení změn v obsahu. Lze také překladač volat vždy dvakrát (lze nastavit v Texmakeru) n.}
    	       
\end{frame}

\begin{frame}[fragile]
    \frametitle{Výhody Texu}
    5. Citace
	
	Odkaz na citaci
    \begin{verbatim}
		\cite{návěstí}
    \end{verbatim}	        	       
		
	
	Vytvořit citaci
	\begin{enumerate}[a)]
		\item pomocí prostředí \textit{thebibliography} 
			
		\item nástrojem \hologo{BibTeX}
		
	\end{enumerate}
	
	Více informací naleznete v dokumentu: \\
	\textit{ITY6 - Vytváření prezentací}
\end{frame}		

\begin{frame}[fragile]
    \frametitle{Výhody Texu}
    6. Poznámky	
    
	Poznámky pod čarou
    \begin{verbatim}
		\footnote{text vaší poznámky} 
    \end{verbatim}	        	       

    \frametitle{Výhody Texu}
    7. Odkazy v textu
	
    \begin{verbatim}
		\label{návěstí} %identifikuje objekt
		\ref{návěstí} %odkaz na číslo objektu
		\pageref{návěští} 
		%odkaz na číslo stránky s objektem
    \end{verbatim}	        	       
\end{frame}



\begin{frame}[fragile]
    \frametitle{Výhody Texu}
    8. Vkládání vzorců (1)
	
	Tex je nejmocnější nástroj na práci se vzorci.	

    \begin{verbatim}
		\begin{eqnarray}
			3x & = & 6(x-9)+15 \\
			3x & = & 6x-54+15 \\
			3x-6x & = & -39 \\
			3x & = & 39 \\
			x & = & 13
		\end{eqnarray}
    \end{verbatim}	        	       
\end{frame}


\begin{frame}[fragile]
    \frametitle{Výhody Texu}
    8. Vkládání vzorců (2)
	
	\begin{eqnarray*}
		3x & = & 6(x-9)+15 \\
		3x & = & 6x-54+15 \\
		3x-6x & = & -39 \\
		3x & = & 39 \\
		x & = & 13
	\end{eqnarray*}
\end{frame}

\begin{frame}[fragile]
    \frametitle{Výhody Texu}
    9. Číslování stran
\end{frame}



\end{document}