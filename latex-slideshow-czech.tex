\documentclass[14pt]{beamer} % beamer - typ/šablona prezentace

\usepackage[czech]{babel} % nastavuje české popisky např. u obsahu, referencí, tabulek, obázků 
\usepackage[utf8]{inputenc} % použito UTF8 kvůli češtině (zvládá prakticky všechny jazyky na světě)
\usepackage[T1]{fontenc}

\usepackage{lmodern}
%\usepackage{datetime}
\usepackage{amssymb} % podpůrná knihovna pro matematické symboly
\usepackage{enumerate} % umožňuje širší možnosti nastavení enumerate

\usepackage{graphicx} % vkládání obrázků
\usepackage{hologo} % logo BibTeX

\usepackage{multicol} % pro praci s více sloupci

% Themes: http://www.hartwork.org/beamer-theme-matrix/
\mode<presentation>{\usetheme{Madrid}}
\usecolortheme{beaver}
\beamertemplatenavigationsymbolsempty 
\setbeamertemplate{title page}[default][colsep=0bp,rounded=true]
\setbeamertemplate{itemize items}{-} %$\circ$
\setbeamercolor*{item}{fg=black}
\setbeamertemplate{enumerate item}[default]

\author{J. Páral, V. Boček}
\institute[paral@robotikabrno.cz]{Pobočka Robotárna - Dům dětí a mládeže Brno, Helceletova\\[0.5cm]}
\title{Jak a proč používat \LaTeX}


\begin{document}

\frame{\titlepage}

\begin{frame}
    \frametitle{Osnova prezentace}
    \begin{center}
		\begin{enumerate}[a)]
			\item Úvod
			\item Jak na to
			\item Výhody \LaTeX{\lower .5ex\hbox {U}}
			\item Slabiny
			\item Prezentace v \LaTeX{\lower .5ex\hbox {U}}
			\item A kam dál?
		\end{enumerate}
    \end{center}
\end{frame}


\begin{frame}
    \frametitle{Úvod}
    \begin{center}
		\begin{enumerate} %[<+->]
			\item Co je to \LaTeX %\LaTeX{\lower .5ex\hbox {U}\kern -.125emM}
				\begin{itemize}
					\item program pro počítačovou sazbu			
				\end{itemize}	
			\item Kdy a proč vznikl
				\begin{itemize}
					\item vznikl v 70. letech 20. století
					\item tvůrce Donald Ervin Knuth
					\item mnoho chyb ve skriptech :-)
				\end{itemize}			
			\item K čemu je primárně určen
				\begin{itemize}
					\item primárně vyvinut pro sázení vzorců
			\end{itemize}
			\item Kdy se vyplatí jej používat
				\begin{itemize}
					\item rozsáhlé a strukturované práce \\
					(ročníkovky, SOČky, bakálářky, diplomky,...)
				\end{itemize}
		\end{enumerate}
    \end{center}
\end{frame}

\begin{frame}
    \frametitle{Jak na to}
	Co potřebuji pro zprovoznění Texu:
    \begin{center}
		\begin{enumerate} %[<+->]
			\item Překládací software
				\begin{itemize}
					\item \href{https://miktex.org/}{TeX Live} [Windows, MacOsX, Linux]
					\item \href{https://miktex.org/}{MiKTex} [Windows]		
				\end{itemize}	
			\item Editor zdrojové kódu
				\begin{itemize}
					\item \href{http://www.xm1math.net/texmaker/}{Texmaker} [free - Windows, MacOsX, Linux]
					\item \href{https://atom.io/}{Atoms} [free - Windows, MacOsX, Linux]
					\item \href{http://www.pspad.com/cz/}{PSPad} [free - Windows]
					\item \href{https://notepad-plus-plus.org/}{Notepad++} [free - Windows]
				\end{itemize}			
		\end{enumerate}
    \end{center}
\end{frame}


\begin{frame}[fragile]
    \frametitle{Jak na to}
	Instalace Linux:

	\begin{verbatim}
	% Debian/Ubuntu/Xubuntu
	sudo apt-get install texlive-full
	\end{verbatim}
	
	
	\begin{verbatim}
	% Fedora
	sudo yum install texlive-scheme-full
	\end{verbatim}

	Překládáme příkazy
	\verb|pdflatex JMENO_SOUBORU.tex|
	
\end{frame}


\begin{frame}[fragile]
    \frametitle{Jak na to}
	Ukázka jednoduchého dokumentu v \LaTeX{\lower .5ex\hbox {U}}:
	\begin{center}
	\begin{verbatim}
		\documentclass[12pt]{article}
		\begin{document}
		Hello world!
		$a^2+b^2=c^2$ %math mode 
		\end{document}
	\end{verbatim}
	\end{center}
	Pozor na češtinu!
	\verb|\usepackage[utf8]{inputenc}|
	
\end{frame}

\begin{frame}[fragile]
    \frametitle{Výhody \LaTeX{\lower .5ex\hbox {U}}}
    \begin{center}
		\begin{enumerate}
			\item Jednotné formátování v dokumentu
			\item Sazba stránek a kapitol
			\item Generování obsahu
			\item Seznamy obrázků, tabulek
			\item Citace
			\item Poznámky 
			\item Odkazy v textu
			\item Vkládání vzorců
			\item Číslování stran	
			\item Odrážky, číslované seznamy
			\item Přenositelnost
		\end{enumerate}
    \end{center}
\end{frame}

\begin{frame}[fragile]
    \frametitle{Výhody \LaTeX{\lower .5ex\hbox {U}}}
    1. Jednotné formátování v dokumentu
    \begin{center}
		\begin{itemize}
			\item používá všude stejný font
			\item a již v základu jsou pěkné :-)
			\item nestaráte se o formátování nadpisů,\\
			odkazů, poznámek, citací
		\end{itemize}
    \end{center}
\end{frame}

\begin{frame}[fragile]
    \frametitle{Výhody \LaTeX{\lower .5ex\hbox {U}}}
    2. Sazba stránek a kapitol
    	     
    Nová stránka:
    \begin{verbatim}
	\newpage % začne na nové stránce
    \end{verbatim}

    Kapitoly:\\   
    \verb|% popis/ukázka|	       
    \verb|\titulek[ text do obsahu ]{ nadpis }|
	\verb|\titulek*{ nadpis } nečíslovaný nadpis|
    
    \begin{verbatim}
		% reálné použití
		\section{Nadpis první úrovně}
		\subsection{Nadpis druhé úrovně}
		\subsubsection{Nadpis třetí úrovně}
    \end{verbatim}
\end{frame}

\begin{frame}[fragile]
    \frametitle{Výhody \LaTeX{\lower .5ex\hbox {U}}}
    3. Generování obsahu
	
	Lze provést přidáním jednoho příkazu a následně se aktualizuje při každém přeložení. 

    \begin{verbatim}
		\tableofcontents
    \end{verbatim}	    

	\textit{Pozor, většinou je potřeba přeložit dokument dvakrát, pro zobrazení změn v obsahu. Nebo lze automatický volat překladač dvakrát.}
\end{frame}

\begin{frame}[fragile]
    \frametitle{Výhody \LaTeX{\lower .5ex\hbox {U}}}
    4. Seznamy obrázků, vzorců, tabulek
	
	Platí to samé jako pro obsah. 

    \begin{verbatim}
		\listoffigures %seznam obrázků
		\listoftables %seznam tabulek
		
    \end{verbatim}	    

	\textit{Pozor, většinou je potřeba přeložit dokument dvakrát, pro zobrazení změn v obsahu. Lze také překladač volat vždy dvakrát (lze nastavit v Texmakeru) n.}
    	       
\end{frame}

\begin{frame}[fragile]
    \frametitle{Výhody \LaTeX{\lower .5ex\hbox {U}}}
    5. Citace
	
	Odkaz na citaci
    \begin{verbatim}
		\cite{návěstí}
    \end{verbatim}	        	       
		
	
	Vytvořit citaci
	\begin{enumerate}[a)]
		\item pomocí prostředí \textit{thebibliography} 
			
		\item nástrojem \hologo{BibTeX}
		
	\end{enumerate}
	
	Více informací naleznete v dokumentu: \\
	\textit{ITY6 - Vytváření prezentací}
\end{frame}		

\begin{frame}[fragile]
    \frametitle{Výhody \LaTeX{\lower .5ex\hbox {U}}}
    6. Poznámky	
    
	Poznámky pod čarou
    \begin{verbatim}
		\footnote{text vaší poznámky} 
    \end{verbatim}	        	       

    \frametitle{Výhody \LaTeX{\lower .5ex\hbox {U}}}
    7. Odkazy v textu
	
    \begin{verbatim}
		\label{návěstí} %identifikuje objekt
		\ref{návěstí} %odkaz na číslo objektu
		\pageref{návěští} 
		%odkaz na číslo stránky s objektem
    \end{verbatim}	        	       
\end{frame}



\begin{frame}[fragile]
    \frametitle{Výhody \LaTeX{\lower .5ex\hbox {U}}}
    8. Vkládání vzorců (1)
	
	\LaTeX{\lower .5ex\hbox {U}} je nejmocnější nástroj na práci se vzorci.	

    \begin{verbatim}
			$a^2+b^2=c^2$ 
			% $ - pro vkládání do textu
	\end{verbatim}	   
    $a^2+b^2=c^2$

    \begin{verbatim}
		\begin{equation*}
		\lim_{x \to \infty}\frac{\sin^2 x + 
		\cos^2 x}{4} = y \nonumber
		\end{equation*}
    \end{verbatim}	        	       
\end{frame}


\begin{frame}[fragile]
    \frametitle{Výhody \LaTeX{\lower .5ex\hbox {U}}}
    8. Vkládání vzorců (2)
	
	\begin{equation*}
	\lim_{x \to \infty}\frac{\sin^2 x + \cos^2 x}{4} = y \nonumber
	\end{equation*}	
	
	Číslované vzorce \verb|\begin{equation}|:\\
	\begin{equation*}
		\int_a^b f(x) \mathrm{d}x = 
		-\int_b^a f(x) \mathrm{d}x 
	\end{equation*}
\end{frame}

\begin{frame}[fragile]
    \frametitle{Výhody \LaTeX{\lower .5ex\hbox {U}}}
    9. Číslování stran \footnote{\href{https://en.wikibooks.org/wiki/LaTeX/Counters}{https://en.wikibooks.org/wiki/LaTeX/Counters}}
    
    Nastavení a zobrazení čítače stránek na \verb|x|: 
    \begin{verbatim}	
	\setcounter{page}{x}
	\end{verbatim}
	
	Nastavení a zobrazení čítače stránek od jedné:
    \begin{verbatim}
	\setcounter{page}{1}
	\end{verbatim}

	Nastavení způsob sazby čísla stránky:
    \begin{verbatim}
	\pagenumbering{styl}
	\end{verbatim}
    
    Nastavení pozice na stránce:
    \begin{verbatim}
	\pagestyle{styl}
	\end{verbatim}

\end{frame}

\begin{frame}[fragile]
    \frametitle{Výhody \LaTeX{\lower .5ex\hbox {U}}}
    10. Odrážky, číslované seznamy (1)
    
    \begin{multicols}{2}
    Odrážky
    \begin{verbatim}
		\begin{itemize}
			\item první odrážka
			\item druhá odrážka
		\end{itemize}
    \end{verbatim}	
    
    \begin{itemize}
		\item první odrážka
		\item druhá odrážka
	\end{itemize}
    
    Číslované seznamy
    \begin{verbatim}
		\begin{enumerate}
			\item první položka
			\item druhá položka
		\end{enumerate}
    \end{verbatim}
		\begin{enumerate}
			\item první položka
			\item druhá položka
		\end{enumerate}

    \end{multicols}
\end{frame}

\begin{frame}[fragile]
    \frametitle{Výhody \LaTeX{\lower .5ex\hbox {U}}}
    10. Odrážky, číslované seznamy (2)
        
    Číslované seznamy -- zanořování
    \begin{verbatim}
		\begin{enumerate}[a)]
			\item první položka
			\begin{enumerate}
				\item první pod položka
				\item druhá pod položka
			\end{enumerate}			
			\item druhá položka
		\end{enumerate}
    \end{verbatim}

\end{frame}

\begin{frame}[fragile]
    \frametitle{Výhody \LaTeX{\lower .5ex\hbox {U}}}
    10. Odrážky, číslované seznamy (3)
    
    Číslované seznamy -- zanořování
		\begin{enumerate}
			\item první položka
			\begin{enumerate}
				\item první pod položka
				\item druhá pod položka
			\end{enumerate}			
			\item druhá položka
		\end{enumerate}
		
	Změna čísel na písmena: \verb|\begin{enumerate}[a)]|	
		\begin{enumerate}[a)]
			\item první položka
			\begin{enumerate}
				\item první pod položka
				\item druhá pod položka
			\end{enumerate}			
			\item druhá položka
		\end{enumerate}		

\end{frame}




\end{document}

	\begin{multicols}{2}
	Lorem ipsum dolor sit amet, consectetur adipiscing elit. Proin tincidunt mollis nisl, in finibus dolor accumsan ultricies. Integer consectetur purus eu lacus sodales consectetur. Donec fermentum lacinia mi, ac malesuada nisi convallis in. Phasellus aliquam quam elit, hendrerit feugiat nunc pulvinar non.
	\end{multicols}